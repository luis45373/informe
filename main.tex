\documentclass[12pt]{article}
\usepackage[utf8]{inputenc}
\usepackage[spanish]{babel}
\usepackage[utf8]{inputenc}
\setlength{\parindent}{0.pt}


\title{La crisis de fundamentos derivo en la computación }
\author{Luis David Muñoz }

\begin{document}
\maketitle
\section{Introducción: }

En el presente escrito se tratará de contextualizar como se empezó a diseñar maquinas programables que eran capaces de resolver ciertos problemas siguiendo una serie de pasos a partir de ideas y postulados hechos por científicos, filósofos y matemáticos tiempos atrás, todos aquellos postulados tenían como principal argumento las matemáticas.

\newpage
En este mundo tal y como lo conocemos sucedió un acontecimiento que marcó un antes y un después, la revolución industrial fue un acontecimiento que evoluciono la economía, la vida social y principalmente la tecnología a finales del siglo XIX, todos estos avances científicos y tecnológicos están basados en las matemáticas.
Todos los estudios realizados sobre este lenguaje matemático dieron paso al infinito un termino que todos los matemáticos de la época no se atrevían a mencionarlo, pero se debía aclarar este termino para dar el gran paso a los avances tecnológicos y principalmente al nacimiento de la computación. 
\\\\
Georg Cantor un matemático alemán fue de los primeros en dar una respuesta al infinito real, cantor se dedico a estudiar los conjuntos, donde finalmente dio cuenta de su teoría que hay muchos infinitos y infinitos más grandes que otros. Cantor dedico toda su vida a defender su descubrimiento siendo criticado por grandes matemáticos de su época como lo fue Leopold kronecker quien le había dado clases a cantor en la universidad de Berlín, Leopold critico fuertemente los estudios de cantor diciendo que en su teoría lo que menos encontraba eran matemáticas, los estudios de cantor llevaron mas adelante a descubrir que las matemáticas tienen limitaciones.
\\\\
Dado al descubrimiento de que las matemáticas tienen limitaciones dio paso a un debate de fundamentos con el fin de solucionar el problema de las limitaciones y llevar las matemáticas a la normalidad, David Hilbert un matemático alemán propuso un programa formalista que se basaba en combinar ideas ya aceptadas para demostrar teoremas nuevos que cumplieran una serie de procesos finitos, pero este programa no tuvo éxito debido a que el matemático austriaco kurt Gödel demostró que con procesos finitos ningún sistema podría ser consistente y completo a la vez. Gracias a esta serie de eventos y descubrimientos se dio paso a la automatización del pensamiento con ello llegaría la programación y las primeras computadoras dando así paso a una era tecnológica.
\\\\
Se generaron muchas discusiones sobre las limitaciones de las matemáticas, pero el gran problema era saber cuales eran esas limitaciones para así poder dar el gran paso a la automatización del pensamiento a través de la computación, este problema fue resuelto por Alan Turing y kurt Gödel quienes combinaron ideas estableciendo los fundamentos y los alcances del razonamiento humano.
\\
Teniendo presente las limitaciones matemáticas, el pensamiento empezó a ser computarizado por maquinas que realizaban ciertas tareas siguiendo un orden de pasos definido, siguiendo con los estudios de Gödel y llevándolos a un nuevo nivel, Alan Turing empezó su diseño de los computadores programables. Principalmente Alan Turing fue quien descifraba mensajes nazis en la segunda guerra mundial a través de una maquina llamada maquina enigma, expertos en el tema aseguran que gracias a las intervenciones de Turing con la maquina enigma en la segunda guerra mundial esta se redujo de entre 2 a 4 años de guerra, gracias a estas intervenciones Turing fue ganando prestigio en la sociedad que le permitió tiempo después relacionarse con grandes científicos de la época como Neuman y Russell.
\\\\
Turing fue el principal pionero de la computación teórica dejando grandes estudios y artículos. Uno de sus principales artículos fue el que publico en 1936, donde definía que era computable y que no lo era. Lo computable era todo aquello que se podía resolver con un algoritmo lo demás era no computable.
En la computación existían problemas que no tenían solución algorítmica, algoritmos que presentarían ciclos infinitos y las computadoras se congelaban y era necesario reiniciarlas, todos esos problemas fueron predichos por Turing.
\\\\
En 1936 un ingeniero alemán Konrad Zuse financiado por el gobierno, fabrico la primera computadora programable considerada por muchos expertos en la actualidad, la Z1 fue una calculadora eléctrica binaria que ocupaba grandes dimensiones de tamaño en comparación con una calculadora en la actualidad, esto dio paso a grandes avances y mejoras respecto a la z1. Finalmente obteniendo la tecnología que hoy por hoy nos permite desde comunicarnos fácilmente hasta el otro lado del mundo hasta maquinas que buscan vida en otras galaxias.

\newpage
\section{Referencias: }
\begin{itemize}
    \item https://www.vix.com/es/btg/curiosidades/4274/historia-de-la-computadora-los-inicios
    
    \item https://www.rsme.es/2013/02/algunos-vinculos-entre-los-teoremas-de-godel-y-turing-en-el-blog-del-ano-turing-en-el-pais/
    
    \item https://users.dcc.uchile.cl/~cgutierr/otros/godel-turing-bits.pdf
    
    \item https://www.bbvaopenmind.com/ciencia/matematicas/asi-termino-el-sueno-de-las-matematicas-infalibles/
    
    \item https://www.bbc.com/mundo/noticias-43568588
    
    \item https://plus.maths.org/content/goumldel-and-limits-logic
    
    \item https://www.youtube.com/watch?v=ntlIA0KwJQ
    
    \item https://www.youtube.com/watch?v=KTUVdXI2vng
    
    \item https://www.autocrecimiento.com/cultura/historia-del-infinito/
    
    \item https://www.lavanguardia.com/economia/20130707/54377301482/la-revolucion-tecnologica-industrial.html
    
    \item https://prezi.com/4zlwjaiojgaq/el-mundo-a-finales-del-siglo-xix-e-inicios-del-siglo-xx/
    
    \item https://es.wikipedia.org/wiki/Revoluci%C3%B3n_Industrial
    
    
    
    
    
    
    
    
    
\end{itemize}







\end{document}
